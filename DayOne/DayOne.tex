% Options for packages loaded elsewhere
\PassOptionsToPackage{unicode}{hyperref}
\PassOptionsToPackage{hyphens}{url}
%
\documentclass[
  ignorenonframetext,
]{beamer}
\usepackage{pgfpages}
\setbeamertemplate{caption}[numbered]
\setbeamertemplate{caption label separator}{: }
\setbeamercolor{caption name}{fg=normal text.fg}
\beamertemplatenavigationsymbolsempty
% Prevent slide breaks in the middle of a paragraph
\widowpenalties 1 10000
\raggedbottom
\setbeamertemplate{part page}{
  \centering
  \begin{beamercolorbox}[sep=16pt,center]{part title}
    \usebeamerfont{part title}\insertpart\par
  \end{beamercolorbox}
}
\setbeamertemplate{section page}{
  \centering
  \begin{beamercolorbox}[sep=12pt,center]{part title}
    \usebeamerfont{section title}\insertsection\par
  \end{beamercolorbox}
}
\setbeamertemplate{subsection page}{
  \centering
  \begin{beamercolorbox}[sep=8pt,center]{part title}
    \usebeamerfont{subsection title}\insertsubsection\par
  \end{beamercolorbox}
}
\AtBeginPart{
  \frame{\partpage}
}
\AtBeginSection{
  \ifbibliography
  \else
    \frame{\sectionpage}
  \fi
}
\AtBeginSubsection{
  \frame{\subsectionpage}
}
\usepackage{lmodern}
\usepackage{amssymb,amsmath}
\usepackage{ifxetex,ifluatex}
\ifnum 0\ifxetex 1\fi\ifluatex 1\fi=0 % if pdftex
  \usepackage[T1]{fontenc}
  \usepackage[utf8]{inputenc}
  \usepackage{textcomp} % provide euro and other symbols
\else % if luatex or xetex
  \usepackage{unicode-math}
  \defaultfontfeatures{Scale=MatchLowercase}
  \defaultfontfeatures[\rmfamily]{Ligatures=TeX,Scale=1}
\fi
\usetheme[]{Berlin}
% Use upquote if available, for straight quotes in verbatim environments
\IfFileExists{upquote.sty}{\usepackage{upquote}}{}
\IfFileExists{microtype.sty}{% use microtype if available
  \usepackage[]{microtype}
  \UseMicrotypeSet[protrusion]{basicmath} % disable protrusion for tt fonts
}{}
\makeatletter
\@ifundefined{KOMAClassName}{% if non-KOMA class
  \IfFileExists{parskip.sty}{%
    \usepackage{parskip}
  }{% else
    \setlength{\parindent}{0pt}
    \setlength{\parskip}{6pt plus 2pt minus 1pt}}
}{% if KOMA class
  \KOMAoptions{parskip=half}}
\makeatother
\usepackage{xcolor}
\IfFileExists{xurl.sty}{\usepackage{xurl}}{} % add URL line breaks if available
\IfFileExists{bookmark.sty}{\usepackage{bookmark}}{\usepackage{hyperref}}
\hypersetup{
  pdftitle={Introduction},
  hidelinks,
  pdfcreator={LaTeX via pandoc}}
\urlstyle{same} % disable monospaced font for URLs
\newif\ifbibliography
\setlength{\emergencystretch}{3em} % prevent overfull lines
\providecommand{\tightlist}{%
  \setlength{\itemsep}{0pt}\setlength{\parskip}{0pt}}
\setcounter{secnumdepth}{-\maxdimen} % remove section numbering

\title{Introduction}
\date{}

\begin{document}
\frame{\titlepage}

\begin{frame}{Plan for the Day}
\protect\hypertarget{plan-for-the-day}{}

\begin{itemize}
\tightlist
\item
  Walk through the syllabus
\item
  Take a pre-test on econ and statistics skills
\item
  Divide up into groups
\end{itemize}

\end{frame}

\begin{frame}{Syllabus}
\protect\hypertarget{syllabus}{}

You can find the syllabus here and any slide here
(\url{https://github.com/woodsjam/Course-Energy-Economics})

\begin{itemize}
\tightlist
\item
  There is also a link in D2L
\item
  Some slides are up there but not all.
\end{itemize}

\end{frame}

\begin{frame}{Energy Economics}
\protect\hypertarget{energy-economics}{}

Economics and structure of energy markets, with a focus on electricity.
Examines current policy issues arising from energy production and use.

\begin{itemize}
\tightlist
\item
  We will be looking at energy markets specifically at the start of
  class, but some of the more interesting topic are in the
  ``consequences of production and use''

  \begin{itemize}
  \tightlist
  \item
    Is Fracking a ``resource curse''?
  \item
    Does development of wind resources have an effect on real estate
    prices?
  \end{itemize}
\item
  The field has a mix of:

  \begin{itemize}
  \tightlist
  \item
    Powerful econometrics
  \item
    Microeconomic theory
  \end{itemize}
\end{itemize}

We will be tuning the class to your level of expertise and filling in
the blanks as we go.

\end{frame}

\begin{frame}{Who Am I?}
\protect\hypertarget{who-am-i}{}

\begin{itemize}
\tightlist
\item
  Ph.D.~Economist, UC Davis
\item
  First generation college.
\item
  Background in energy conservation and program evaluation consulting
  going back to the early 90s.

  \begin{itemize}
  \tightlist
  \item
    Projects were multidisciplinary. Rarely did a project not involve,
    economists, engineers and others.
  \item
    ETO external review committee
  \item
    I have steel toed boots.
  \end{itemize}
\item
  Former:

  \begin{itemize}
  \tightlist
  \item
    Oregon HECC commissioner
  \item
    Parkrose School Board Chair
  \item
    Multiple non-profit boards
  \item
    County Budget Committees
  \item
    Legislative Candidate.
  \end{itemize}
\item
  Hobbies: Painfully Stereotypical Portlander
\end{itemize}

\end{frame}

\begin{frame}{Warnings}
\protect\hypertarget{warnings}{}

\begin{itemize}
\tightlist
\item
  I'm Dyslexic. If you see a slide without a typo -- be amazed.
\item
  Don't think that you have to understand everything

  \begin{itemize}
  \tightlist
  \item
    You should read everything.
  \item
    I usually read things over and over again and learn new things every
    time.
  \item
    If you understand half -- amazing.
  \item
    The more you read, the easier it is to learn more. It really builds
    on itself.
  \end{itemize}
\end{itemize}

\end{frame}

\begin{frame}{Warnings (Con't)}
\protect\hypertarget{warnings-cont}{}

\begin{itemize}
\tightlist
\item
  Don't be afraid to Google a picture or an article to help you
  understand something in class.

  \begin{itemize}
  \tightlist
  \item
    You need to learn how to help yourself.
  \item
    Learned helplessness is a pet peeve of mine.
  \end{itemize}
\item
  Do not wait till the last minute

  \begin{itemize}
  \tightlist
  \item
    Don't make an optimistic, tightly scheduled plans.
  \item
    No plan survives contact with reality
  \item
    Start everything as early as possible.
  \end{itemize}
\end{itemize}

\end{frame}

\begin{frame}{Where can you take this knowledge}
\protect\hypertarget{where-can-you-take-this-knowledge}{}

\begin{itemize}
\tightlist
\item
  Energy Efficiency evaluation

  \begin{itemize}
  \tightlist
  \item
    Plenty of local firms do this (Many former students)
  \item
    Energy Trust of Oregon (ETO)
  \end{itemize}
\item
  Utilities

  \begin{itemize}
  \tightlist
  \item
    Northwest Natural
  \item
    PGE, Pacificorp
  \end{itemize}
\item
  Government

  \begin{itemize}
  \tightlist
  \item
    Oregon department of energy
  \item
    PUC
  \item
    Many opportunities as Cap and Trade takes off
  \end{itemize}
\item
  Other For Profit

  \begin{itemize}
  \tightlist
  \item
    Anything related to home automation.
  \item
    Use off high frequency meter data.
  \item
    Carbon Offset certification and evaluation.
  \end{itemize}
\end{itemize}

\end{frame}

\begin{frame}{Look at Course Requirements and the Syllabus}
\protect\hypertarget{look-at-course-requirements-and-the-syllabus}{}

A few things of note

\begin{itemize}
\tightlist
\item
  There is a term paper.
\item
  Reading discussion with some outside prep.
\item
  My contention is that if you are good with theoretical models or
  empirical models or explaining them to others, you are employable.

  \begin{itemize}
  \tightlist
  \item
    The last one is frequently forgotten.
  \item
    They are often the highest fliers in the industry.
  \end{itemize}
\end{itemize}

\end{frame}

\begin{frame}{Prerequisites}
\protect\hypertarget{prerequisites}{}

\begin{itemize}
\tightlist
\item
  Undergrad Students: EC 311/415
\item
  Graduate Students: NONE except graduate standing
\end{itemize}

The usual pattern is the undergrads are econ majors and know 201
material and some 311. Grad students tend to be engineers with less econ
but more math than the undergrads.

\end{frame}

\begin{frame}{Contact Information}
\protect\hypertarget{contact-information}{}

\begin{itemize}
\tightlist
\item
  My office is in CH 241-O.
\item
  Drop in office hours are Monday 3:00 PM - 4:00 through the last week
  of class. There is no need to make an appointment for these hours --
  just come.
\item
  If you can't attend regular office hours, please check my calendar
  \url{https://woodsj.youcanbook.me/}. I will make a limited number of
  20 minute slots available each week. If you make an appointment and
  fail to show up without first canceling, I will penalize your final
  exam score two points.
\end{itemize}

Meetings can be via phone or even the hangout/meet function in your
pdx.edu email.

I will also keep Slack up during office hours to answer questions.

\end{frame}

\begin{frame}{Slack}
\protect\hypertarget{slack}{}

I avoid email since it is a trash heap and a constant distraction.

\begin{itemize}
\tightlist
\item
  The class will use slack \url{https://psuenergyecon.slack.com}
\item
  Handles direct messages, forum style responses.
\item
  You should be able to sign up with your pdx.edu account
\item
  Can access via webpage, apps in Android and iOS.
\item
  Linked in d2l.
\item
  Officially supported PSU software
\item
  Common in industry
\item
  Many of you already use it.
\end{itemize}

\end{frame}

\begin{frame}{Hints on Slack}
\protect\hypertarget{hints-on-slack}{}

\begin{itemize}
\tightlist
\item
  Prefer \#general to direct messaging me.

  \begin{itemize}
  \tightlist
  \item
    Questions about course material is for \#general
  \item
    Questions about your grade or anything private is for @woodsj
  \end{itemize}
\item
  ``I can't find it anywhere'' is usually answered with:

  \begin{itemize}
  \tightlist
  \item
    A link to a google search.
  \end{itemize}
\item
  Don't start and end with, ``I got 7 but the key shows 245. What did I
  do wrong?''

  \begin{itemize}
  \tightlist
  \item
    Show your work. The fix is often a single parameter.
  \end{itemize}
\item
  Provide full text of the question you are working on.
\end{itemize}

In short, make it easy for someone to help you.

\end{frame}

\begin{frame}{More Slack Hints}
\protect\hypertarget{more-slack-hints}{}

\begin{itemize}
\tightlist
\item
  Rants and bellyaching put you at the bottom of the queue or off the
  queue.
\item
  Night before a deadline? You are on your own.
\item
  Set personal Do Not Disturb times. It it defaults to 10pm -8am
\end{itemize}

\end{frame}

\begin{frame}{Textbook}
\protect\hypertarget{textbook}{}

\end{frame}

\begin{frame}{Adding Material As We Go}
\protect\hypertarget{adding-material-as-we-go}{}

Course has not been taught since 2016 because of low enrollment.

\begin{itemize}
\tightlist
\item
  Add newer articles to existing, but many in the syllabus are classics
  and very readable.
\item
  Hot topics in Oregon:

  \begin{itemize}
  \tightlist
  \item
    Transportation Electrification
  \item
    Joining the CAISO
  \item
    Cap and Trade
  \end{itemize}
\end{itemize}

\end{frame}

\begin{frame}{What a Class Meeting Looks Like}
\protect\hypertarget{what-a-class-meeting-looks-like}{}

\begin{itemize}
\tightlist
\item
  Framing lecture that is no more than an hour.
\item
  Break into discussion groups.
\item
  Questions randomly given out
\item
  Each group has a scribe that will summarize discussion in a shared
  google doc for the class
\item
  I bounce from group to group pushing conversation to make sure y'all
  have actual read the material
\item
  Share out and take questions from other groups.
\item
  Future readings
\end{itemize}

\end{frame}

\begin{frame}{Mix-in options}
\protect\hypertarget{mix-in-options}{}

\begin{itemize}
\tightlist
\item
  I give out a few questions ahead of time
\item
  I give out specific questions for groups to prep and present. Longer
  lead time on this so you can meet.
\end{itemize}

\end{frame}

\begin{frame}{Course Structure}
\protect\hypertarget{course-structure}{}

\begin{itemize}
\tightlist
\item
  Introductory reading in energy covering Coal, Natural Gas, Electricity
  and Oil.
\item
  The big push is some work on your own in an area that interests you.

  \begin{itemize}
  \tightlist
  \item
    Supports on how to write a research paper.
  \item
    Encouragement to browse/read widely to find an interesting paper
    topic.
  \item
    Create a narrowly framed abstract
  \item
    Cut it down more after meeting with me.
  \item
    Create an annotated bibliography, basically an outline with
    citations and how you will use them in the paper
  \item
    Draft Paper
  \item
    Final paper with separate document explaining how you addressed my
    comments on your draft.
  \end{itemize}
\end{itemize}

\end{frame}

\begin{frame}{The Term Paper}
\protect\hypertarget{the-term-paper}{}

\begin{itemize}
\tightlist
\item
  Literature review or empirical paper
\item
  Lit Reviews are single author.
\item
  Empirical, which means econometrics and data handling, may have up to
  three authors.
\item
  Keep it under 20 pages not including bibliography.
\end{itemize}

The intent is to be able to take it to a conference or use as evidence
that you can write when looking for a job.

\end{frame}

\begin{frame}{Help?}
\protect\hypertarget{help}{}

\begin{itemize}
\tightlist
\item
  Library guy will give a talk on the how-to part of a literature
  review.
\item
  Handout from me.
\item
  Syllabus has links to major journals and working paper archives that
  cover energy.
\item
  A few hint books. McCloskey for all and one chapter from Wooldridge's
  econometrics book for those doing empirical papers.
\end{itemize}

\end{frame}

\begin{frame}{Questions}
\protect\hypertarget{questions}{}

\end{frame}

\begin{frame}{Lets see where you are}
\protect\hypertarget{lets-see-where-you-are}{}

Time to take the pre-test.

\begin{itemize}
\tightlist
\item
  Does not count for your grade
\item
  Don't freak out if you can't do all of it.
\item
  They are pitched a little high.
\end{itemize}

I'm trying to sort out how to dial in lecture and where I need to add
skills.

\end{frame}

\begin{frame}{Lets Break Into-Groups}
\protect\hypertarget{lets-break-into-groups}{}

\end{frame}

\begin{frame}{Reading for Wednesday}
\protect\hypertarget{reading-for-wednesday}{}

\begin{itemize}
\tightlist
\item
  Read Dahl Ch 1-2. It is about 40 pages.
\item
  ``Energy Primer: A Handbook of Energy Market Basics'', Ch 1. About 3
  pages
\item
  DON'T PANIC
\end{itemize}

\end{frame}

\begin{frame}{How to Read Academic Papers}
\protect\hypertarget{how-to-read-academic-papers}{}

It is a multipass system.

\begin{itemize}
\tightlist
\item
  Read the Title and abstract (Papers)
\item
  Read the section headings
\item
  Read the introduction
\item
  Read the conclusion
\item
  Look at the figures
\item
  Look at the equations
\item
  Skim the whole thing
\item
  Read it but don't freak out if you don't get everything.

  \begin{itemize}
  \tightlist
  \item
    Keep notes on your reactions, how the topic connects back and
    questions (My personal notes on papers are filled with profanity)
  \item
    Repeat as time allows.
  \end{itemize}
\end{itemize}

You, and the author, will never fully understand the paper/chapter.

\end{frame}

\end{document}
